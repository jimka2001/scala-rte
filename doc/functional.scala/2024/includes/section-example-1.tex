\begin{frame}[t]{Example 1: How does a pattern predicate work?}
  \only<1-2>{\tt \textcolor{blue}{[13 2.0 6.0 4 "a" "an" "the" -5 2.0 3.0 4.0 7 8.0]}}%
  \only<3>{\tt [13 2.0 6.0 4 "a" "an" "the" -5 2.0 3.0 4.0 7 8.0]}%
  \only<4>{\tt [\colorbox{orange!30}{\Large 13} 2.0 6.0 4 "a" "an" "the" -5 2.0 3.0 4.0 7 8.0]}%
  \only<5>{\tt [13 \colorbox{orange!30}{\Large 2.0} 6.0 4 "a" "an" "the" -5 2.0 3.0 4.0 7 8.0]}%
  \only<6>{\tt [13 2.0 \colorbox{orange!30}{\Large 6.0} 4 "a" "an" "the" -5 2.0 3.0 4.0 7 8.0]}%
  \only<7>{\tt [13 2.0 6.0 \colorbox{orange!30}{\Large 4} "a" "an" "the" -5 2.0 3.0 4.0 7 8.0]}%
  \only<8>{\tt [13 2.0 6.0 4 \colorbox{orange!30}{\Large "a"} "an" "the" -5 2.0 3.0 4.0 7 8.0]}%
  \only<9>{\tt [13 2.0 6.0 4 "a" \colorbox{orange!30}{\Large "an"} "the" -5 2.0 3.0 4.0 7 8.0]}%
  \only<10>{\tt [13 2.0 6.0 4 "a" "an" \colorbox{orange!30}{\Large "the"} -5 2.0 3.0 4.0 7 8.0]}%
  \only<11>{\tt [13 2.0 6.0 4 "a" "an" "the" \colorbox{orange!30}{\Large -5} 2.0 3.0 4.0 7 8.0]}%
  \only<12>{\tt [13 2.0 6.0 4 "a" "an" "the" -5 \colorbox{orange!30}{\Large 2.0} 3.0 4.0 7 8.0]}%
  \only<13>{\tt [13 2.0 6.0 4 "a" "an" "the" -5 2.0 \colorbox{orange!30}{\Large 3.0} 4.0 7 8.0]}%
  \only<14>{\tt [13 2.0 6.0 4 "a" "an" "the" -5 2.0 3.0 \colorbox{orange!30}{\Large 4.0} 7 8.0]}%
  \only<15>{\tt [13 2.0 6.0 4 "a" "an" "the" -5 2.0 3.0 4.0 \colorbox{orange!30}{\Large 7} 8.0]}%
  \only<16>{\tt [13 2.0 6.0 4 "a" "an" "the" -5 2.0 3.0 4.0 7 \colorbox{orange!30}{\Large 8.0}]}%
  \only<17>{\tt [13 2.0 6.0 4 "a" "an" "the" -5 2.0 3.0 4.0 7 8.0]}%
 \only<18>{\textcolor{blue}{Decision procedure is $O(n)$,\Emph{independent of syntactical complexity} of the RTE.}}%

 \begin{columns}
   \begin{column}{0.4\textwidth}
     \only<2>{

       \bigskip

       Does the sequence match the pattern?
       \textcolor{greeny}{${(Int \cdot ( Double^{+} ~\vee String^{+}))^{+}}$}

       \bigskip

     
       We \textcolor{blue}{construct~a finite automaton} (DFA).
       
       \bigskip
       
       \Challenge{2} How to construct a finite automaton from an RTE?}%
     \only<3-16>{\textcolor{greeny}{\LARGE ${(Int \cdot ( Double^{+} ~\vee String^{+}))^{+}}$}}\leavevmode\\[2cm]%
     \only<3-16>{\textcolor{orange}{\LARGE Does it match?}}%
     \only<17>{\textcolor{greeny}{\LARGE Yes, it's a match!}}%
     \only<18>{\includegraphics[height=3.cm]{exploding-head}}
   \end{column}
   \begin{column}{0.6\textwidth}
     \only<2,18>{\scalebox{0.95}{\input{fig-3}}}%
     \only<3>{\scalebox{0.95}{\input{fig-3-0}}}%
     \only<4>{\scalebox{0.95}{\documentclass{standalone}
  \usepackage{tikz}
  \usetikzlibrary{arrows.meta, automata, bending, positioning, shapes.misc}
  \tikzstyle{automaton}=[shorten >=1pt, >={Stealth[bend,round]}, initial text=]

\begin{document}
\begin{tikzpicture}[automaton, auto, thick]
  \node[state,initial,rounded rectangle,fill=yellow] (0) {$0$};
  \node[state,rounded rectangle,fill=yellow] (1) [right=20mm of 0] {$1$};
  \node[state,accepting,thick,rounded rectangle] (2) [above right=7mm and 30mm of 1] {$2$};
  \node[state,accepting,thick,rounded rectangle] (3) [below right=7mm and 30mm of 1] {$3$};
  \path[->] (0) edge[orange, line width=3] node {$Int$} (1);
  \path[->] (1) edge[bend left=15]  node[pos=.8] {$Double$} (2);
  \path[->] (1) edge[bend right=15] node[swap] {$String$} (3);
  \path[->] (2) edge[bend left=15]  node {$Int$} (1);
  \path[->] (2) edge[loop above]    node {$Double$} (2);
  \path[->] (3) edge[bend right=15] node[swap] {$Int$} (1);
  \path[->] (3) edge[loop below]    node {$String$} (3);
\end{tikzpicture}
\end{document}
}}%
     \only<5>{\scalebox{0.95}{\input{fig-3-1-double}}}%
     \only<6>{\scalebox{0.95}{\input{fig-3-2-double}}}%
     \only<7>{\scalebox{0.95}{\input{fig-3-2-int}}}%
     \only<8>{\scalebox{0.95}{\input{fig-3-1-String}}}%
     \only<9>{\scalebox{0.95}{\input{fig-3-3-String}}}%
     \only<10>{\scalebox{0.95}{\input{fig-3-3-String}}}%
     \only<11>{\scalebox{0.95}{\input{fig-3-3-int}}}%
     \only<12>{\scalebox{0.95}{\input{fig-3-1-double}}}%
     \only<13>{\scalebox{0.95}{\input{fig-3-2-double}}}%
     \only<14>{\scalebox{0.95}{\input{fig-3-2-double}}}%
     \only<15>{\scalebox{0.95}{\input{fig-3-2-int}}}%
     \only<16>{\scalebox{0.95}{\input{fig-3-1-double}}}%
     \only<17>{\scalebox{0.95}{\input{fig-3-2}}}%
   \end{column}
 \end{columns}
\end{frame}



\begin{frame}{Deterministic (DFA) vs Non-deterministic (NFA)}

  Suppose sequence = \code{[2,  \colorbox{orange!30}{\Large 3}, 5.6F]}

  \begin{columns}[T]
    \begin{column}{0.3\textwidth}
      \centering
      
      \begin{align*}
        Int&\subseteq Number\\
        Int &\cap Number \neq \emptyset
      \end{align*}%
      \scalebox{0.8}{\input{tikz-int-float}}%
    \end{column}%
    \begin{column}{0.7\textwidth}
      \only<1,2>{\scalebox{0.95}{\documentclass{standalone}
  \usepackage{tikz}
  \usetikzlibrary{arrows.meta, automata, bending, positioning, shapes.misc}
  \tikzstyle{automaton}=[shorten >=1pt, >={Stealth[bend,round]}, initial text=]

\begin{document}
\begin{tikzpicture}[automaton, auto]
  \node[state,initial,rounded rectangle] (0) {$0$};
  \node[state,rounded rectangle] (1) [right=10mm of 0] {$1$};
  \node[state,rounded rectangle] (2) [above right=7mm and 30mm of 1] {$2$};
  \node[state,rounded rectangle] (3) [below right=7mm and 30mm of 1] {$3$};
  \path[->] (0) edge node {$Int$} (1);
  \path[->] (1) edge[orange, line width=3pt, bend left=15]  node[pos=.8] {$Int$} (2);
  \path[->] (1) edge[orange, line width=3pt, bend right=15] node[swap] {$Number$} (3);
\end{tikzpicture}
\end{document}
}}
      
      \only<2>{\includegraphics[height=2cm]{red-head}\LARGE Backtracking?}%
      \only<3>{\scalebox{0.95}{\documentclass{standalone}
  \usepackage{tikz}
  \usetikzlibrary{arrows.meta, automata, bending, positioning, shapes.misc}
  \tikzstyle{automaton}=[shorten >=1pt, >={Stealth[bend,round]}, initial text=]

\begin{document}
\begin{tikzpicture}[automaton, auto, thick]
  \node[state,initial,rounded rectangle] (0) {$0$};
  \node[state,rounded rectangle] (1) [right=10mm of 0] {$1$};
  \node[state,rounded rectangle] (2) [above right=7mm and 20mm of 1] {$2$};
  \node[state,accepting] (3) [right=20mm of 2] {$3$};
  \node[state,rounded rectangle] (4) [below right=7mm and 20mm of 1] {$4$};
  \node[state,accepting] (5) [right=20mm of 4] {$5$};
  \path[->] (0) edge node {$Int$} (1);
  \path[->] (1) edge[color=deterministic, line width=3pt, bend left=15]  node[pos=.8] {$Int$} (2);
  \path[->] (2) edge  node {$String$} (3);
  \path[->] (1) edge[color=deterministic, line width=3pt, bend right=15] node[swap] {$Number \cap \overline{Int}$} (4);
  \path[->] (4) edge  node {$Float$} (5);
  \path[->] (2) edge  node {$Float$} (5);
\end{tikzpicture}
\end{document}
}}%
      \only<4-6>{\scalebox{0.65}{\documentclass{standalone}
  \usepackage{tikz}
  \usetikzlibrary{arrows.meta, automata, bending, positioning, shapes.misc}
  \tikzstyle{automaton}=[shorten >=1pt, >={Stealth[bend,round]}, initial text=]

\begin{document}
\begin{tikzpicture}[automaton, auto, thick]
  \node[state,initial,rounded rectangle] (0) {$0$};
  \node[state,rounded rectangle] (1) [right=10mm of 0] {$1$};
  \node[state,rounded rectangle] (2) [above right=7mm and 20mm of 1] {$2$};
  \node[state,accepting] (3) [right=20mm of 2] {$3$};
  \node[state,rounded rectangle] (4) [below right=7mm and 20mm of 1] {$4$};
  \node[state,accepting] (5) [right=20mm of 4] {$5$};
  \path[->] (0) edge node {$Int$} (1);
  \path[->] (1) edge[color=deterministic, line width=3pt, bend left=15]  node[pos=.8] {$Int$} (2);
  \path[->] (2) edge  node {$String$} (3);
  \path[->] (1) edge[color=deterministic, line width=3pt, bend right=15] node[swap] {$Number \cap \overline{Int}$} (4);
  \path[->] (4) edge  node {$Float$} (5);
  \path[->] (2) edge  node {$Float$} (5);
\end{tikzpicture}
\end{document}
}}%
      
      \only<4-6>{\Challenge{3} How to support \textcolor{greeny}{$Int\cap\overline{Number}$} in Scala?\\}
      \only<5,6>{\Challenge{4} How to partition types?\\}
      \only<6>{\Challenge{5} How to avoid duplicate type checks?}%

      \only<5>{
        \Eg, type decomposition\\\quad\textcolor{greeny}{$\{String,Int,Number\}\to\{String,Int,Number\cap\overline{Int}\}$}
      }%
    \end{column}
  \end{columns}
\end{frame}

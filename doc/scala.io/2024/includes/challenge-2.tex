\section{DFA Construction}

{  \setbeamercolor{background canvas}{bg=chaptercolor}
\begin{frame}{Challenge \textnumero 2}{DFA Construction}

  Given an RTE, generate a finite automaton.

  \begin{itemize}
  \item Well-known techniques exists to construct DFAs from RE
  \item Adapt them to work with RTEs
  \item See \emph{Type-Checking of Heterogeneous Sequences in Common Lisp},
    \url{http://www.lrde.epita.fr/dload/papers/newton.16.els.pdf}
  \item Library: \code{xymbolyco}
  \item Enforce determinism
  \end{itemize}
\end{frame}

}

\begin{frame}{DFA}{What are Finite Automata?}
  \scalebox{0.75}{\input{fig-td}}
  \begin{itemize}
  \item   Every RTE corresponds to a Deterministic Finite Automaton (DFA), \code{Rte.toDfa(rte)}, \Emph{$O(2^n)$}.

  \item   The DFA recognizes a sequence matching the RTE, \Emph{$O(n)$}.

  \end{itemize}
\end{frame}



\begin{frame}{Demo}{Sample Flow}
  \centering
   \includegraphics[height=0.8\textheight]{demo.png}
\end{frame}


\begin{frame}{Demo}{Sample Flow}
  Decide whether two RTEs are equivalent, and if not what is the difference.

  \code{ScalaIo2024.scala}

  \begin{itemize}
  \item define data
  \item define types and RTEs
  \item define two candidate patterns
  \item both match data
  \item show DFAs
  \item show xor DFA
  \item show spanning trace
  \end{itemize}

\end{frame}




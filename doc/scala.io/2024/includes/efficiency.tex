\section{Efficient Redundant Type Checks}

\begin{frame}{Efficient Redundant Type Checks}
\end{frame}


\newsavebox\typecaseAbox
\begin{lrbox}{\typecaseAbox}
  \begin{minipage}{8cm}
\begin{lstlisting}[style=scalaioScala]
  val N = SAtomic(classOf[Number])
  val I = SAtomic(classOf[Int])

  if SAnd(N,!I).typep(x)
    Some(1)
  else if I.typep(x)
    Some(2)
  else if (!N).typep(x)
    Some(3)
  else
    None
\end{lstlisting}
  \end{minipage}
\end{lrbox}

\newsavebox\typecaseBbox
\begin{lrbox}{\typecaseBbox}
  \begin{minipage}{8cm}
\begin{lstlisting}[style=scalaioScala]
  if N.typep(x) {
    if SAnd(N,!I).typep(x)
      Some(1)
    else if I.typep(x)
      Some(2)
    else if (!N).typep(x)
      Some(3)
    else      None
  } else {
    if SAnd(N,!I).typep(x)
      Some(1)
    else if I.typep(x)
      Some(2)
    else if (!N).typep(x)
      Some(3)
    else      None
  }
\end{lstlisting}
  \end{minipage}
\end{lrbox}

\newsavebox\typecaseCbox
\begin{lrbox}{\typecaseCbox}
  \begin{minipage}{6cm}
\begin{lstlisting}[style=scalaioScala]
  if N.typep(x) {
    if SAnd(STop,!I).typep(x)
      Some(1)
    else if I.typep(x)
      Some(2)
    else if (!STop).typep(x)
      Some(3)
    else      None
  } else {
    if SAnd(SEmpty,!SEmpty).typep(x)
      Some(1)
    else if SEmpty.typep(x)
      Some(2)
    else if (!SEmpty).typep(x)
      Some(3)
    else      None
  }
\end{lstlisting}
  \end{minipage}
\end{lrbox}

\newsavebox\typecaseDbox
\begin{lrbox}{\typecaseDbox}
  \begin{minipage}{8cm}
\begin{lstlisting}[style=scalaioScala]
  if N.typep(x) {
    if (!I).typep(x)
      Some(1)
    else if I.typep(x)
      Some(2)
    else if SEmpty.typep(x)
      Some(3)
    else      None
  } else {
    if SEmpty.typep(x)
      Some(1)
    else if SEmpty.typep(x)
      Some(2)
    else if STop.typep(x)
      Some(3)
    else      None
  }
\end{lstlisting}
  \end{minipage}
\end{lrbox}

\newsavebox\typecaseEbox
\begin{lrbox}{\typecaseEbox}
  \begin{minipage}{8cm}
\begin{lstlisting}[style=scalaioScala]
  if N.typep(x) {
    if (!I).typep(x)
      Some(1)
    else if I.typep(x)
      Some(2)
    else if false
      Some(3)
    else      None
  } else {
    if false
      Some(1)
    else if false
      Some(2)
    else if true
      Some(3)
    else      None
  }
\end{lstlisting}
  \end{minipage}
\end{lrbox}

\newsavebox\typecaseFbox
\begin{lrbox}{\typecaseFbox}
  \begin{minipage}{8cm}
\begin{lstlisting}[style=scalaioScala]
  if N.typep(x) {
    if (!I).typep(x)
      Some(1)
    else if I.typep(x)
      Some(2)
    else      None
  } else Some(3)
\end{lstlisting}
  \end{minipage}
\end{lrbox}

\newsavebox\typecaseGbox
\begin{lrbox}{\typecaseGbox}
  \begin{minipage}{8cm}
\begin{lstlisting}[style=scalaioScala]
  if N.typep(x) {
    if I.typep(x) {
      if (!I).typep(x)
        Some(1)
      else if I.typep(x)
        Some(2)
      else None
    } else {
      if (!I).typep(x)
        Some(1)
      else if I.typep(x)
        Some(2)
      else      None
    }
  } else Some(3)
\end{lstlisting}
  \end{minipage}
\end{lrbox}

\newsavebox\typecaseHbox
\begin{lrbox}{\typecaseHbox}
  \begin{minipage}{8cm}
\begin{lstlisting}[style=scalaioScala]
  if N.typep(x) {
    if I.typep(x) {
      if (!STop).typep(x)
        Some(1)
      else if STop.typep(x)
        Some(2)
      else None
    } else {
      if (!SEmpty).typep(x)
        Some(1)
      else if SEmpty.typep(x)
        Some(2)
      else None
    }
  } else Some(3)
\end{lstlisting}
  \end{minipage}
\end{lrbox}

\newsavebox\typecaseIbox
\begin{lrbox}{\typecaseIbox}
  \begin{minipage}{8cm}
\begin{lstlisting}[style=scalaioScala]
  if N.typep(x) {
    if I.typep(x) {
      if SEmpty.typep(x)
        Some(1)
      else if STop.typep(x)
        Some(2)
      else      None
    } else {
      if STop.typep(x)
        Some(1)
      else if SEmpty.typep(x)
        Some(2)
      else      None
    }
  } else Some(3)
\end{lstlisting}
  \end{minipage}
\end{lrbox}

\newsavebox\typecaseJbox
\begin{lrbox}{\typecaseJbox}
  \begin{minipage}{8cm}
\begin{lstlisting}[style=scalaioScala]
  if N.typep(x) {
    if I.typep(x) {
      if false
        Some(1)
      else if true
        Some(2)
      else      None
    } else {
      if true
        Some(1)
      else if false
        Some(2)
      else      None
    }
  } else Some(3)
\end{lstlisting}
  \end{minipage}
\end{lrbox}

\newsavebox\typecaseKbox
\begin{lrbox}{\typecaseKbox}
  \begin{minipage}{8cm}
\begin{lstlisting}[style=scalaioScala]
  if N.typep(x) {
    if I.typep(x)
      Some(2)
    else
      Some(1)
  } else Some(3)
\end{lstlisting}
  \end{minipage}
\end{lrbox}



\begin{frame}{Sequential Type Check}
  In general a state in the DFA may have several transitions, each with its own type label.
  \begin{columns}
    \begin{column}{0.5\textwidth}
      \usebox\typecaseAbox
    \end{column}
    \begin{column}{0.5\textwidth}  %%
      \scalebox{0.9}{\documentclass{standalone}
  \usepackage{tikz}
  \usetikzlibrary{arrows.meta, automata, bending, positioning, shapes.misc}
  \tikzstyle{automaton}=[shorten >=1pt, >={Stealth[bend,round]}, initial text=]

\begin{document}
\begin{tikzpicture}[automaton, auto]
  \node[state,initial,rounded rectangle] (0) {$0$};
  \node[state,accepting, rounded rectangle] (2) [right=30mm of 0] {$2$};
  \node[state,accepting, rounded rectangle] (1) [above right=7mm and 30mm of 0] {$1$};
  \node[state,accepting, rounded rectangle] (3) [below right=7mm and 30mm of 0] {$3$};
  \path[->] (0) edge node {$Int$} (2);
  \path[->] (0) edge[bend left=15]  node[pos=.8] {$Number~ \cap~ !Int$} (1);
  \path[->] (0) edge[bend right=15] node[swap] {$!~Number$} (3);
\end{tikzpicture}
\end{document}
}
    \end{column}    
  \end{columns}

  Some types may be checked multiple times.  We can rewrite the code to eliminate redundant checks.

\end{frame}


\begin{frame}{Rewrite: 1}
  Introduce \code{if N.typep(x) ... else ...}.

  \begin{columns}
    \begin{column}{0.5\textwidth}
      \usebox\typecaseAbox
    \end{column}
    \begin{column}{0.5\textwidth}  %%
      \usebox\typecaseBbox
    \end{column}    
  \end{columns}
\end{frame}

\begin{frame}{Rewrite: 2}
  In \code{if} part, replace supertypes of \code{N} with \code{STop}.  $Int\subset Number$
  
  In \code{else} part, replace subtypes of \code{N} with \code{SEmpty}

  \begin{columns}
    \begin{column}{0.5\textwidth}
      \usebox\typecaseBbox
    \end{column}
    \begin{column}{0.5\textwidth}  %%
      \usebox\typecaseCbox
    \end{column}    
  \end{columns}
\end{frame}

\begin{frame}{Rewrite: 3}
  \begin{columns}
    \begin{column}{0.5\textwidth}
      \code{SAnd(STop,x)} $\to$ \code{x}\\
      \code{!STop} $\to$ \code{SEmpty}
      
      \usebox\typecaseCbox
    \end{column}
    \begin{column}{0.5\textwidth}  %%
      \code{SAnd(SEmpty,x)} $\to$ \code{SEmpty}\\
      \code{!SEmpty} $\to$ \code{STop}

      \usebox\typecaseDbox
    \end{column}    
  \end{columns}
\end{frame}

\begin{frame}{Rewrite: 4}
  \begin{columns}
    \begin{column}{0.5\textwidth}
      \code{SEmpty.typep(x)} $\to$ \code{false}
      
      \usebox\typecaseDbox
    \end{column}
    \begin{column}{0.5\textwidth}  %%
      \code{STop.typep(x)} $\to$ \code{true}

      \usebox\typecaseEbox
    \end{column}    
  \end{columns}
\end{frame}

\begin{frame}{Rewrite: 5}
  \code{(if true x else y)} $\to$ \code{x}\\
  \code{(if false x else y)} $\to$ \code{y}

  \begin{columns}
    \begin{column}{0.5\textwidth}
      \usebox\typecaseEbox
    \end{column}
    \begin{column}{0.5\textwidth}  %%
      \usebox\typecaseFbox
    \end{column}    
  \end{columns}
\end{frame}

\begin{frame}{Rewrite: 6}
  Introduce \code{if I.typep(x) ... else ...}.

  \begin{columns}
    \begin{column}{0.5\textwidth}
      \usebox\typecaseFbox
    \end{column}
    \begin{column}{0.5\textwidth}  %%
      \usebox\typecaseGbox
    \end{column}    
  \end{columns}
\end{frame}

\begin{frame}{Rewrite: 7}
  In \code{if} part, replace supertypes of \code{I} with \code{STop} and subtypes in \code{else} part with \code{SEmpty}.

  \begin{columns}
    \begin{column}{0.5\textwidth}
      \usebox\typecaseGbox
    \end{column}
    \begin{column}{0.5\textwidth}  %%
      \usebox\typecaseHbox
    \end{column}    
  \end{columns}
\end{frame}


\begin{frame}{Rewrite: 8}

  \begin{columns}
    \begin{column}{0.5\textwidth}
      \code{!STop} $\to$ \code{SEmpty}
      \usebox\typecaseHbox
    \end{column}
    \begin{column}{0.5\textwidth}  %%
      \code{!SEmpty} $\to$ \code{STop}
      \usebox\typecaseIbox
    \end{column}    
  \end{columns}
\end{frame}

\begin{frame}{Rewrite: 9}

  \begin{columns}
    \begin{column}{0.5\textwidth}
      \code{SEmpty.typep(x)} $\to$ \code{false}
      \usebox\typecaseIbox
    \end{column}
    \begin{column}{0.5\textwidth}  %%
      \code{STop.typep(x)} $\to$ \code{true}
      \usebox\typecaseJbox
    \end{column}    
  \end{columns}
\end{frame}

\begin{frame}{Rewrite: 10}
  \code{(if true x else y)} $\to$ \code{x}\\
  \code{(if false x else y)} $\to$ \code{y}

  \begin{columns}
    \begin{column}{0.5\textwidth}
      \usebox\typecaseJbox
    \end{column}
    \begin{column}{0.5\textwidth}  %%
      \usebox\typecaseKbox
    \end{column}
  \end{columns}
\end{frame}


\begin{frame}{Rewrite: Summary}
  Code has been rewritten so that each type check occurs no more than once.

  \begin{columns}
    \begin{column}{0.5\textwidth}
      \usebox\typecaseAbox
    \end{column}
    \begin{column}{0.5\textwidth}  %%
      \usebox\typecaseKbox
    \end{column}
  \end{columns}

  And it is clear the the code never returns \code{None}.

\end{frame}



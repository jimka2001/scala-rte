\subsection{Efficiency: Redundant Type Checks}

{  %% chapter slide
  \setbeamercolor{background canvas}{bg=sectioncolor}

\begin{frame}{\Challenge{5} Redundant Type Check}
  Select correct transition, avoiding \Emph{redundant type checks}.

  \medskip

  \scalebox{0.85}{\documentclass{standalone}
  \usepackage{tikz}
  \usetikzlibrary{arrows.meta, automata, bending, positioning, shapes.misc}
  \tikzstyle{automaton}=[shorten >=1pt, >={Stealth[bend,round]}, initial text=]

\begin{document}
\begin{tikzpicture}[automaton, auto]
  \node[state,initial,rounded rectangle] (0) {$0$};
  \node[state,accepting, rounded rectangle] (2) [right=30mm of 0] {$2$};
  \node[state,accepting, rounded rectangle] (1) [above right=7mm and 30mm of 0] {$1$};
  \node[state,accepting, rounded rectangle] (3) [below right=7mm and 30mm of 0] {$3$};
  \path[->] (0) edge node {$Int$} (2);
  \path[->] (0) edge[bend left=15]  node[pos=.8] {$Number~ \cap~ !Int$} (1);
  \path[->] (0) edge[bend right=15] node[swap] {$!~Number$} (3);
\end{tikzpicture}
\end{document}
}
\end{frame}
}


\newsavebox\typecaseAbox
\begin{lrbox}{\typecaseAbox}
  \begin{minipage}{8cm}
    %% dont re-indent this file
\begin{lstlisting}[style=scalaioScala]
val N = SAtomic(classOf[Number])
val I = SAtomic(classOf[Int])

if (N & !I).typep(x)
  Some(1)
else if I.typep(x)
  Some(2)
else if (!N).typep(x)
  Some(3)
else
  None
\end{lstlisting}

  \end{minipage}
\end{lrbox}

\newsavebox\typecaseITEbox
\begin{lrbox}{\typecaseITEbox}
  \begin{minipage}{8cm}
    %% dont re-indent this file
\begin{lstlisting}[style=scalaioScala]
[N & !I, Some(1),
         [I, Some(2),
             [!N, Some(3),
                  None]]]
\end{lstlisting}

  \end{minipage}
\end{lrbox}

\newsavebox\typecaseITEafterbox
\begin{lrbox}{\typecaseITEafterbox}
  \begin{minipage}{8cm}
%% dont re-indent this file
\begin{lstlisting}[style=scalaioScala]
[N, [I, Some(2),
        Some(1)],
    Some(3)]
\end{lstlisting}

  \end{minipage}
\end{lrbox}

\newsavebox\typecaseBabox
\begin{lrbox}{\typecaseBabox}
  \begin{minipage}{8cm}
%% dont re-indent this file
\begin{lstlisting}[style=scalaioScala]
if ~~N.typep(x)~~ {
  ... original code ...
} ~~ELSE~~ {
  ... original code ...
}
\end{lstlisting}

  \end{minipage}
\end{lrbox}

\newsavebox\typecaseBaabox
\begin{lrbox}{\typecaseBaabox}
  \begin{minipage}{8cm}
%% dont re-indent this file
\begin{lstlisting}[style=scalaioScala]
if ~~N.typep(x)~~ {
  if (N & !I).typep(x)
    Some(1)
  else if I.typep(x)
    Some(2)
  else if (!N).typep(x)
    Some(3)
  else      None
} ~~ELSE~~ {
  if (N & !I).typep(x)
    Some(1)
  else if I.typep(x)
    Some(2)
  else if (!N).typep(x)
    Some(3)
  else      None
}
\end{lstlisting}

  \end{minipage}
\end{lrbox}

\newsavebox\typecaseBbox
\begin{lrbox}{\typecaseBbox}
  \begin{minipage}{8cm}
    %% dont re-indent this file
\begin{lstlisting}[style=scalaioScala]
if N.typep(x) {
  if (N & !I).typep(x)
    Some(1)
  else if I.typep(x)
    Some(2)
  else if (!N).typep(x)
    Some(3)
  else      None
} else {
  if (N & !I).typep(x)
    Some(1)
  else if I.typep(x)
    Some(2)
  else if (!N).typep(x)
    Some(3)
  else      None
}
\end{lstlisting}

  \end{minipage}
\end{lrbox}

\newsavebox\typecaseCbox
\begin{lrbox}{\typecaseCbox}
  \begin{minipage}{6cm}
%% dont re-indent this file
\begin{lstlisting}[style=scalaioScala]
if N.typep(x) {
  if (STop & !I).typep(x)
    Some(1)
  else if I.typep(x)
    Some(2)
  else if (!STop).typep(x)
    Some(3)
  else      None
} else {
  if (SEmpty & !SEmpty).typep(x)
    Some(1)
  else if SEmpty.typep(x)
    Some(2)
  else if (!SEmpty).typep(x)
    Some(3)
  else None
}
\end{lstlisting}

  \end{minipage}
\end{lrbox}

\newsavebox\typecaseChbox
\begin{lrbox}{\typecaseChbox}
  \begin{minipage}{6cm}
%% dont re-indent this file
\begin{lstlisting}[style=scalaioScala]
if N.typep(x) {
  if (~~STop~~ & !I).typep(x)
    Some(1)
  else if I.typep(x)
    Some(2)
  else if (!~~STop~~).typep(x)
    Some(3)
  else      None
} else {
  if (~~SEmpty~~ & !~~SEmpty~~).typep(x)
    Some(1)
  else if ~~SEmpty~~.typep(x)
    Some(2)
  else if (!~~SEmpty~~).typep(x)
    Some(3)
  else None
}
\end{lstlisting}

  \end{minipage}
\end{lrbox}

\newsavebox\typecaseDbox
\begin{lrbox}{\typecaseDbox}
  \begin{minipage}{8cm}
    %% dont re-indent this file
\begin{lstlisting}[style=scalaioScala]
if N.typep(x) {
  if (!I).typep(x)
    Some(1)
  else if I.typep(x)
    Some(2)
  else if SEmpty.typep(x)
    Some(3)
  else None
} else {
  if SEmpty.typep(x)
    Some(1)
  else if SEmpty.typep(x)
    Some(2)
  else if STop.typep(x)
    Some(3)
  else None
}
\end{lstlisting}

  \end{minipage}
\end{lrbox}

\newsavebox\typecaseDhbox
\begin{lrbox}{\typecaseDhbox}
  \begin{minipage}{8cm}
    %% dont re-indent this file
\begin{lstlisting}[style=scalaioScala]
if N.typep(x) {
  if (~~!I~~).typep(x)
    Some(1)
  else if I.typep(x)
    Some(2)
  else if ~~SEmpty~~.typep(x)
    Some(3)
  else None
} else {
  if ~~SEmpty~~.typep(x)
    Some(1)
  else if SEmpty.typep(x)
    Some(2)
  else if ~~STop~~.typep(x)
    Some(3)
  else None
}
\end{lstlisting}

  \end{minipage}
\end{lrbox}

\newsavebox\typecaseEbox
\begin{lrbox}{\typecaseEbox}
  \begin{minipage}{8cm}
    %% dont re-indent this file
\begin{lstlisting}[style=scalaioScala]
if N.typep(x) {
  if (!I).typep(x)
    Some(1)
  else if I.typep(x)
    Some(2)
  else if false
    Some(3)
  else      None
} else {
  if false
    Some(1)
  else if false
    Some(2)
  else if true
    Some(3)
  else      None
}
\end{lstlisting}

  \end{minipage}
\end{lrbox}

\newsavebox\typecaseEhbox
\begin{lrbox}{\typecaseEhbox}
  \begin{minipage}{8cm}
    %% dont re-indent this file
\begin{lstlisting}[style=scalaioScala]
if N.typep(x) {
  if (!I).typep(x)
    Some(1)
  else if I.typep(x)
    Some(2)
  else if ~~FALSE~~
    Some(3)
  else      None
} else {
  if false
    Some(1)
  else if ~~FALSE~~
    Some(2)
  else if ~~TRUE~~
    Some(3)
  else      None
}
\end{lstlisting}

  \end{minipage}
\end{lrbox}

\newsavebox\typecaseFbox
\begin{lrbox}{\typecaseFbox}
  \begin{minipage}{8cm}
    %% dont re-indent this file
\begin{lstlisting}[style=scalaioScala]
if N.typep(x) {
  if (!I).typep(x)
    Some(1)
  else if I.typep(x)
    Some(2)
  else      None
} else Some(3)
\end{lstlisting}

  \end{minipage}
\end{lrbox}

\newsavebox\typecaseFhbox
\begin{lrbox}{\typecaseFhbox}
  \begin{minipage}{8cm}
    %% dont re-indent this file
\begin{lstlisting}[style=scalaioScala]
if N.typep(x) {
  if (!I).typep(x)
    Some(1)
  else if I.typep(x)
    Some(2)
  else ~~None~~
} else ~~Some(3)~~
\end{lstlisting}

  \end{minipage}
\end{lrbox}

\newsavebox\typecaseGabox
\begin{lrbox}{\typecaseGabox}
  \begin{minipage}{8cm}
%% dont re-indent this file
\begin{lstlisting}[style=scalaioScala]
if N.typep(x) {
  if ~~I.typep(x)~~ {
    ... original code...
  } ~~ELSE~~ {
    ... original code...
  }
} else Some(3)
\end{lstlisting}

  \end{minipage}
\end{lrbox}

\newsavebox\typecaseGbox
\begin{lrbox}{\typecaseGbox}
  \begin{minipage}{8cm}
    %% dont re-indent this file
\begin{lstlisting}[style=scalaioScala]
if N.typep(x) {
  if I.typep(x) {
    if (!I).typep(x)
      Some(1)
    else if I.typep(x)
      Some(2)
    else None
  } else {
    if (!I).typep(x)
      Some(1)
    else if I.typep(x)
      Some(2)
    else      None
  }
} else Some(3)
\end{lstlisting}

  \end{minipage}
\end{lrbox}

\newsavebox\typecaseGhbox
\begin{lrbox}{\typecaseGhbox}
  \begin{minipage}{8cm}
    %% dont re-indent this file
\begin{lstlisting}[style=scalaioScala]
if N.typep(x) {
  if ~~I.typep(x)~~ {
    if (!I).typep(x)
      Some(1)
    else if I.typep(x)
      Some(2)
    else None
  } ~~ELSE~~ {
    if (!I).typep(x)
      Some(1)
    else if I.typep(x)
      Some(2)
    else      None
  }
} else Some(3)
\end{lstlisting}

  \end{minipage}
\end{lrbox}


\newsavebox\typecaseHbox
\begin{lrbox}{\typecaseHbox}
  \begin{minipage}{8cm}
    %% dont re-indent this file
\begin{lstlisting}[style=scalaioScala]
if N.typep(x) {
  if I.typep(x) {
    if (!STop).typep(x)
      Some(1)
    else if STop.typep(x)
      Some(2)
    else None
  } else {
    if (!SEmpty).typep(x)
      Some(1)
    else if SEmpty.typep(x)
      Some(2)
    else None
  }
} else Some(3)
\end{lstlisting}

  \end{minipage}
\end{lrbox}

\newsavebox\typecaseHhbox
\begin{lrbox}{\typecaseHhbox}
  \begin{minipage}{8cm}
    %% dont re-indent this file
\begin{lstlisting}[style=scalaioScala]
if N.typep(x) {
  if I.typep(x) {
    if (!~~STop~~).typep(x)
      Some(1)
    else if ~~STop~~.typep(x)
      Some(2)
    else None
  } else {
    if (!~~SEmpty~~).typep(x)
      Some(1)
    else if ~~SEmpty~~.typep(x)
      Some(2)
    else None
  }
} else Some(3)
\end{lstlisting}

  \end{minipage}
\end{lrbox}

\newsavebox\typecaseIbox
\begin{lrbox}{\typecaseIbox}
  \begin{minipage}{8cm}
    %% dont re-indent this file
\begin{lstlisting}[style=scalaioScala]
if N.typep(x) {
  if I.typep(x) {
    if SEmpty.typep(x)
      Some(1)
    else if STop.typep(x)
      Some(2)
    else      None
  } else {
    if STop.typep(x)
      Some(1)
    else if SEmpty.typep(x)
      Some(2)
    else      None
  }
} else Some(3)
\end{lstlisting}

  \end{minipage}
\end{lrbox}

\newsavebox\typecaseIhbox
\begin{lrbox}{\typecaseIhbox}
  \begin{minipage}{8cm}
    %% dont re-indent this file
\begin{lstlisting}[style=scalaioScala]
if N.typep(x) {
  if I.typep(x) {
    if ~~SEmpty~~.typep(x)
      Some(1)
    else if STop.typep(x)
      Some(2)
    else      None
  } else {
    if ~~STop~~.typep(x)
      Some(1)
    else if SEmpty.typep(x)
      Some(2)
    else      None
  }
} else Some(3)
\end{lstlisting}

  \end{minipage}
\end{lrbox}

\newsavebox\typecaseJbox
\begin{lrbox}{\typecaseJbox}
  \begin{minipage}{8cm}
    %% dont re-indent this file
\begin{lstlisting}[style=scalaioScala]
if N.typep(x) {
  if I.typep(x) {
    if false
      Some(1)
    else if true
      Some(2)
    else      None
  } else {
    if true
      Some(1)
    else if false
      Some(2)
    else      None
  }
} else Some(3)
\end{lstlisting}

  \end{minipage}
\end{lrbox}

\newsavebox\typecaseJhbox
\begin{lrbox}{\typecaseJhbox}
  \begin{minipage}{8cm}
    %% dont re-indent this file
\begin{lstlisting}[style=scalaioScala]
if N.typep(x) {
  if I.typep(x) {
    if ~~FALSE~~
      Some(1)
    else if ~~TRUE~~
      Some(2)
    else      None
  } else {
    if true
      Some(1)
    else if ~~FALSE~~
      Some(2)
    else      None
  }
} else Some(3)
\end{lstlisting}

  \end{minipage}
\end{lrbox}

\newsavebox\typecaseKbox
\begin{lrbox}{\typecaseKbox}
  \begin{minipage}{8cm}
    %% dont re-indent this file
\begin{lstlisting}[style=scalaioScala]
if N.typep(x) {
  if I.typep(x)
    Some(2)
  else
    Some(1)
} else Some(3)
\end{lstlisting}

  \end{minipage}
\end{lrbox}


\newsavebox\typecaseKhbox
\begin{lrbox}{\typecaseKhbox}
  \begin{minipage}{8cm}
    %% dont re-indent this file
\begin{lstlisting}[style=scalaioScala]
if N.typep(x) {
  if I.typep(x)
    ~~Some(2)~~
  else
    ~~Some(1)~~
} else Some(3)
\end{lstlisting}

  \end{minipage}
\end{lrbox}



\begin{frame}{Sequential Type Check}
  A DFA state may have several \Emph{disjoint} transitions, each with its own type label.
  \begin{columns}
    \begin{column}{0.5\textwidth}
      \usebox\typecaseAbox
    \end{column}
    \begin{column}{0.5\textwidth}  %%
      \scalebox{0.9}{\documentclass{standalone}
  \usepackage{tikz}
  \usetikzlibrary{arrows.meta, automata, bending, positioning, shapes.misc}
  \tikzstyle{automaton}=[shorten >=1pt, >={Stealth[bend,round]}, initial text=]

\begin{document}
\begin{tikzpicture}[automaton, auto]
  \node[state,initial,rounded rectangle] (0) {$0$};
  \node[state,accepting, rounded rectangle] (2) [right=30mm of 0] {$2$};
  \node[state,accepting, rounded rectangle] (1) [above right=7mm and 30mm of 0] {$1$};
  \node[state,accepting, rounded rectangle] (3) [below right=7mm and 30mm of 0] {$3$};
  \path[->] (0) edge node {$Int$} (2);
  \path[->] (0) edge[bend left=15]  node[pos=.8] {$Number~ \cap~ !Int$} (1);
  \path[->] (0) edge[bend right=15] node[swap] {$!~Number$} (3);
\end{tikzpicture}
\end{document}
}
    \end{column}    
  \end{columns}

  Some types may be checked multiple times.  We can rewrite the code to \Emph{eliminate redundant checks}.
\end{frame}

\begin{frame}{Decision Tree Structure}
  We programmatically manipulate \code{if ... else ...} using a lazy, \code{Ite} (if/then/else) data structure similar to the following.

  \begin{columns}
    \begin{column}{0.5\textwidth}
      \usebox\typecaseAbox
    \end{column}
    \begin{column}{0.5\textwidth}  %%
      \usebox\typecaseITEbox
    \end{column}    
  \end{columns}

  For this presentation, we represent the decision tree as \Emph{human readable} Scala code.
\end{frame}

\begin{frame}{Decision Tree, Before and After}
  Viewing the \code{if ... else ...} before and after as decision trees.

  \medskip
  
  Rewrite: $1 \to 2\to 3\to 4\to 5\to 6\to 7\to 8\to 9\to 10$

  \medskip
  
  \begin{columns}
    \begin{column}{0.5\textwidth}
      \usebox\typecaseITEbox
    \end{column}
    \begin{column}{0.5\textwidth}  %%
      \usebox\typecaseITEafterbox
    \end{column}
  \end{columns}
\end{frame}


%% Thanks to John Wickerson, https://tex.stackexchange.com/users/25356/john-wickerson
%% for this arrow
%% https://tex.stackexchange.com/questions/113228/how-can-i-add-a-big-arrow
\def\myLeftArrow{\smash{
  \begin{tikzpicture}[baseline=-2mm]
    \useasboundingbox (-2,0);
    \node[single arrow,draw=black,fill=red!10,minimum width=5cm,minimum height=7cm,shape border rotate=180] at (0,-1) {};
  \end{tikzpicture}
}}


\begin{frame}{Rewrite: $\colorbox{orange!30}{\Huge 1}\to 2\to 3\to 4\to 5\to 6\to 7\to 8\to 9\to 10$}%1
  Introduce \colorbox{pink!30}{\code{if N.typep(x) ... else ...}}

  \begin{columns}
    \begin{column}{0.5\textwidth}
      \only<1,2,3>{\usebox\typecaseAbox}%
      \only<5,6>{\usebox\typecaseBbox}
    \end{column}
    \begin{column}{0.5\textwidth}  %%
      \only<2>{\usebox\typecaseBabox}%
      \only<3>{\usebox\typecaseBaabox}%
      \only<4>{\usebox\typecaseBbox}%
      \only<5>{\myLeftArrow\vfill}
    \end{column}    
  \end{columns}
\end{frame}

\begin{frame}{Rewrite: $1\to \colorbox{orange!30}{\Huge 2}\to 3\to 4\to 5\to 6\to 7\to 8\to 9\to 10$}
  \begin{tabular}{ll}
  In \code{then} part: \colorbox{pink!30}{Supertypes of \code{N} $\to$ \code{STop}}. &
  In \code{else} part: \colorbox{pink!30}{Subtypes of \code{N} $\to$ \code{SEmpty}}.
  \end{tabular}
  
  \begin{columns}
    \begin{column}{0.5\textwidth}
      \only<1,2>{\usebox\typecaseBbox}%
      \only<4,5>{\usebox\typecaseCbox}
    \end{column}
    \begin{column}{0.5\textwidth}  %%
      \only<2,3>{\usebox\typecaseChbox}
      \only<4>{\myLeftArrow\vfill}
    \end{column}    
  \end{columns}
\end{frame}

\begin{frame}{Rewrite: $1\to 2\to \colorbox{orange!30}{\Huge 3}\to 4\to 5\to 6\to 7\to 8\to 9\to 10$}
  \begin{tabular}{ll}
    \colorbox{pink!30}{\code{(STop \& x)} $\to$ \code{x}} &       \colorbox{pink!30}{\code{(SEmpty \& x)} $\to$ \code{SEmpty}}\\
    \colorbox{pink!30}{\code{!STop} $\to$ \code{SEmpty}} &       \colorbox{pink!30}{\code{!SEmpty} $\to$ \code{STop}}    
  \end{tabular}
  \begin{columns}
    \begin{column}{0.5\textwidth}
      \only<1,2>{\usebox\typecaseCbox}%
      \only<4>{\usebox\typecaseDhbox}%
      \only<5>{\usebox\typecaseDbox}
    \end{column}
    \begin{column}{0.5\textwidth}  %%
      \only<2,3>{\usebox\typecaseDhbox}%
      \only<4>{\myLeftArrow\vfill}
    \end{column}    
  \end{columns}
\end{frame}

\begin{frame}{Rewrite: $1\to 2\to 3\to \colorbox{orange!30}{\Huge 4}\to 5\to 6\to 7\to 8\to 9\to 10$}
  \begin{tabular}{ll}
    \colorbox{pink!30}{\code{SEmpty.typep(x)} $\to$ \code{false}} & 
    \colorbox{pink!30}{\code{STop.typep(x)} $\to$ \code{true}}
  \end{tabular}
  \begin{columns}
    \begin{column}{0.5\textwidth}      
      \only<1,2>{\usebox\typecaseDbox}%
      \only<4>{\usebox\typecaseEhbox}%
      \only<5>{\usebox\typecaseEbox}
    \end{column}
    \begin{column}{0.5\textwidth}  %%
      \only<2,3>{\usebox\typecaseEhbox}%
      \only<4>{\myLeftArrow\vfill}
    \end{column}    
  \end{columns}
\end{frame}

\begin{frame}{Rewrite: $1\to 2\to 3\to 4\to \colorbox{orange!30}{\Huge 5}\to 6\to 7\to 8\to 9\to 10$}
  \begin{tabular}{ll}
    \colorbox{pink!30}{\code{(if true x else y)} $\to$ \code{x}} &     
    \colorbox{pink!30}{\code{(if false x else y)} $\to$ \code{y}}
  \end{tabular}
  \begin{columns}
    \begin{column}{0.5\textwidth}
      \only<1,2>{\usebox\typecaseEbox}%
      \only<4>{\usebox\typecaseFhbox}%
      \only<5>{\usebox\typecaseFbox}
    \end{column}
    \begin{column}{0.5\textwidth}  %%
      \only<2,3>{\usebox\typecaseFhbox}%
      \only<4>{\myLeftArrow\vfill}
    \end{column}    
  \end{columns}
\end{frame}

\begin{frame}{Rewrite: $1\to 2\to 3\to 4\to 5\to \colorbox{orange!30}{\Huge 6}\to 7\to 8\to 9\to 10$}
  Introduce \colorbox{pink!30}{\code{if I.typep(x) ... else ...}}

  \begin{columns}
    \begin{column}{0.5\textwidth}
      \only<1,2,3>{\usebox\typecaseFbox}%
      \only<5,6>{\usebox\typecaseGbox}
    \end{column}

    \begin{column}{0.5\textwidth}  %%
      \only<2>{\usebox\typecaseGabox}
      \only<3>{\usebox\typecaseGhbox}%
      \only<4>{\usebox\typecaseGbox}%
      \only<5>{\myLeftArrow\vfill}
    \end{column}    
  \end{columns}
\end{frame}

\begin{frame}{Rewrite: $1\to 2\to 3\to 4\to 5\to 6\to \colorbox{orange!30}{\Huge 7}\to 8\to 9\to 10$}
  \begin{tabular}{ll}
  In \code{then} part: \colorbox{pink!30}{Supertypes of \code{I} $\to$ \code{STop}}.&
  In \code{else} part: \colorbox{pink!30}{Subtypes of \code{I} $\to$ \code{SEmpty}}.
  \end{tabular}

  \begin{columns}
    \begin{column}{0.5\textwidth}
      \only<1,2>{\usebox\typecaseGbox}%
      \only<4,5>{\usebox\typecaseHbox}
    \end{column}
    \begin{column}{0.5\textwidth}  %%
      \only<2,3>{\usebox\typecaseHhbox}%
      \only<4>{\myLeftArrow\vfill}
    \end{column}    
  \end{columns}
\end{frame}


\begin{frame}{Rewrite: $1\to 2\to 3\to 4\to 5\to 6\to 7\to \colorbox{orange!30}{\Huge 8}\to 9\to 10$}
  \begin{tabular}{ll}
      \colorbox{pink!30}{\code{!STop} $\to$ \code{SEmpty}} &    
      \colorbox{pink!30}{\code{!SEmpty} $\to$ \code{STop}}
  \end{tabular}
  \begin{columns}
    \begin{column}{0.5\textwidth}
      \only<1,2>{\usebox\typecaseHbox}%
      \only<4,5>{\usebox\typecaseIbox}
    \end{column}
    \begin{column}{0.5\textwidth}  %%
      \only<2,3>{\usebox\typecaseIhbox}%
      \only<4>{\myLeftArrow\vfill}
    \end{column}    
  \end{columns}
\end{frame}

\begin{frame}{Rewrite: $1\to 2\to 3\to 4\to 5\to 6\to 7\to 8\to \colorbox{orange!30}{\Huge 9}\to 10$}
  \begin{tabular}{ll}
    \colorbox{pink!30}{\code{SEmpty.typep(x)} $\to$ \code{false}} &
    \colorbox{pink!30}{\code{STop.typep(x)} $\to$ \code{true}}  
  \end{tabular}
  \begin{columns}
    \begin{column}{0.5\textwidth}
      \only<1,2>{\usebox\typecaseIbox}%
      \only<4,5>{\usebox\typecaseJbox}
    \end{column}
    \begin{column}{0.5\textwidth}  %%
      \only<2,3>{\usebox\typecaseJhbox}%
      \only<4>{\myLeftArrow\vfill}
    \end{column}    
  \end{columns}
\end{frame}

\begin{frame}{Rewrite: $1\to 2\to 3\to 4\to 5\to 6\to 7\to 8\to 9\to \colorbox{orange!30}{\Huge 10}$}
  \begin{tabular}{ll}
  \colorbox{pink!30}{\code{(if true x else y)} $\to$ \code{x}}
& \colorbox{pink!30}{\code{(if false x else y)} $\to$ \code{y}}
  \end{tabular}
  \begin{columns}
    \begin{column}{0.5\textwidth}
      \only<1,2>{\usebox\typecaseJbox}%
      \only<4,5>{\usebox\typecaseKbox}
    \end{column}
    \begin{column}{0.5\textwidth}  %%
      \only<2,3>{\usebox\typecaseKhbox}%
      \only<4>{\myLeftArrow\vfill}
    \end{column}
  \end{columns}
\end{frame}


\begin{frame}{Rewrite: Summary}
  Code has been rewritten so that \Emph{any type check occurs no more than once}.

  \begin{columns}
    \begin{column}{0.5\textwidth}
      \usebox\typecaseAbox
    \end{column}
    \begin{column}{0.5\textwidth}  %%
      \usebox\typecaseKbox
    \end{column}
  \end{columns}

  And it is clear the the code never returns \code{None}.

\end{frame}


{  %\setbeamercolor{background canvas}{bg=chaptercolor}
\begin{frame}{Challenges of the Project}
  \begin{itemize}
  \item \Challenge{~1}\textcolor{blue}{RTE Representation}:   Representing an RTE in Scala?
  \item \Challenge{~2}\textcolor{blue}{DFA Construction}:  Constucting from RTE?
  \item \Challenge{~3}\textcolor{blue}{Type Lattice}: Union, intersection, complement types?
  \item \Challenge{~4}\textcolor{blue}{Determinism}: Type partitioning?
  \item \Challenge{~5}\textcolor{blue}{Efficiency}:  Avoiding redundant type checks at run-time?
  \end{itemize}
\end{frame}
}





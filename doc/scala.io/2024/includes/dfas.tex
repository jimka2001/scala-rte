


\begin{frame}{Deterministic Finite Automata}
  \centering
  %%By Photographer: Blake Griffin - Finite Automata and Dom Price, CC BY-SA 3.0, https://en.wikipedia.org/w/index.php?curid=47875021
  \includegraphics[width=0.7\textwidth]{fa-band.jpg}
\end{frame}


\begin{frame}{DFA}{What are Finite Automata?}
  \begin{itemize}
  \item   Every RTE corresponds to a Deterministic Finite Automata (DFA), \code{Rte.toDfa(rte)} (theoretically exponential time)

  \item   The DFA will recognize any sequence matching the RTE (in linear time).

  \end{itemize}
\end{frame}

\begin{frame}{\code{SimpleTypeD} Transitions}

  \scalebox{0.7}{\input{fig-td}}

  A DFA transition is labeled by a \code{SimpleTypeD}.
\end{frame}



\begin{frame}{Uses of DFA}
  DFAs can be combined by an algebra.

  \begin{itemize}
  \item intersect RTE languages via intersecting the DFAs
  \item determine equivalence (or subset-ness) of two RTEs
  \item determine habitation and vacuity
  \item extract RTE from DFA
  \end{itemize}
\end{frame}


\begin{frame}{Demo}{Sample Flow}
  Decide whether two RTEs are equivalent, and if not what is the difference.

  \code{ScalaIo2024.scala}

  \begin{itemize}
  \item define data
  \item define types and RTEs
  \item define two candidate patterns
  \item both match data
  \item show DFAs
  \item show xor DFA
  \item show spanning trace
  \end{itemize}

\end{frame}


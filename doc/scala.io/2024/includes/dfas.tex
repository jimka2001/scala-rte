\section{Deterministic Finite Automata}

\begin{frame}{Deterministic Finite Automata}
\end{frame}


\begin{frame}{DFA}{What are Finite Automata?}
  \begin{itemize}
  \item   Every RTE corresponds to a Deterministic Finite Automata (DFA), \code{Rte.toDfa(rte)} (theoretically exponential time)

  \item   The DFA will recognize any sequence matching the RTE (in linear time).

  \end{itemize}
\end{frame}

\begin{frame}{Indeterminant Transitions}

  \scalebox{0.7}{\input{fig-3}}

   A DFA has transitions labeled by \emph{types}.
\end{frame}

\begin{frame}{Indeterminant Transitions}

  \scalebox{0.8}{\documentclass{standalone}
  \usepackage{tikz}
  \usetikzlibrary{arrows.meta, automata, bending, positioning, shapes.misc}
  \tikzstyle{automaton}=[shorten >=1pt, >={Stealth[bend,round]}, initial text=]
  \tikzstyle{accepting}=[double]

\begin{document}
\begin{tikzpicture}[automaton, auto]
  \node[state,color=green,text=black,initial,rounded rectangle] (0) {$0$};
  \node[state,color=red,text=black,rounded rectangle] (1) [right=30mm of 0] {$1$};
  \node[state,color=red,text=black,accepting,rounded rectangle] (2) [above right=7mm and 30mm of 1] {$2$};
  \node[state,color=red,text=black,accepting,rounded rectangle] (3) [below right=7mm and 30mm of 1] {$3$};
  \path[->] (0) edge[color=red, line width=3pt] node {Int \& Odd} (1);
  \path[->] (1) edge[bend left=15]  node[pos=.8] {Int} (2);
  \path[->] (1) edge[bend right=15] node[swap] {String} (3);
  \path[->] (2) edge[bend left=15]  node {Int} (1);
  \path[->] (2) edge[loop above]    node {Int} (2);
  \path[->] (3) edge[bend right=15] node[swap] {Int} (1);
  \path[->] (3) edge[loop below]    node {String} (3);
\end{tikzpicture}
\end{document}
}
  \begin{itemize}

  \item If we can determine that a type is uninhabited, we may eliminate the
  transition and unreachable states.

  \item Subtype question is unanswerable, so some transitions are \emph{indeterminant}.
  \end{itemize}
\end{frame}


\begin{frame}{Indeterminant Transitions}

  \scalebox{0.8}{\documentclass{standalone}
  \usepackage{tikz}
  \usetikzlibrary{arrows.meta, automata, bending, positioning, shapes.misc}
  \tikzstyle{automaton}=[shorten >=1pt, >={Stealth[bend,round]}, initial text=]
  \tikzstyle{accepting}=[double]

\begin{document}
\begin{tikzpicture}[automaton, auto]
  \node[state,color=green,text=black,initial,rounded rectangle] (0) {$0$};
  \node[state,color=red,text=black,rounded rectangle] (1) [right=30mm of 0] {$1$};
  \node[state,color=red,text=black,accepting,rounded rectangle] (2) [above right=7mm and 30mm of 1] {$2$};
  \node[state,color=red,text=black,accepting,rounded rectangle] (3) [below right=7mm and 30mm of 1] {$3$};
  \path[->] (0) edge[color=red, line width=3pt] node {Int \& Odd} (1);
  \path[->] (1) edge[bend left=15]  node[pos=.8] {Int} (2);
  \path[->] (1) edge[bend right=15] node[swap] {String} (3);
  \path[->] (2) edge[bend left=15]  node {Int} (1);
  \path[->] (2) edge[loop above]    node {Int} (2);
  \path[->] (3) edge[bend right=15] node[swap] {Int} (1);
  \path[->] (3) edge[loop below]    node {String} (3);
\end{tikzpicture}
\end{document}
}
  \begin{itemize}

    \item At run-time, given an object, we can always determine type membership.

    \item DFAs with indeterminant transitions nevertheless match sequences in linear time.  
  \end{itemize}
\end{frame}


\begin{frame}{Uses of DFA}
  DFAs can be combined by an algebra.

  \begin{itemize}
  \item intersect RTE languages via intersecting the DFAs
  \item determine equivalence (or subset-ness) of two RTEs
  \item determine habitation and vacuity
  \item extract RTE from DFA
  \end{itemize}
\end{frame}

\section{Deterministic Finite Automata}

\begin{frame}{Deterministic Finite Automata}
\end{frame}


\begin{frame}{DFA}{What are Finite Automata?}

  Every RTE corresponds to a Deterministic Finite Automata (DFA).

  The method \code{Rte.toDfa(...)} creates a DFA which will recognize any sequence matching the RTE.

  Matching time is linear in sequence length, and independent of complexity of the RTE.

  However, DFA generation is (theoretically) exponential in time time.

\end{frame}

\begin{frame}{Indeterminant Transitions}
  A DFA generated from an RTE has transitions labeled by \emph{types} as per SETS.

  If we can determine at construction time that a type is uninhabited, we may eliminate the
  transition, and potentially eliminate states which are not reachable.

  However, since the subtype question is unanswerable, we may have transitions labeled by types
  for which we cannot answer the habitation question.  These are called indeterminant transistions.

  An indeterminant transition may be accessible or non-accessible.

  At run-time, given an object, we can always determine type membership.

  Thus even DFAs with indeterminant transitions still match sequences in linear time.  
\end{frame}


\begin{frame}{Uses of DFA}
  DFAs can be combined by an algebra.

  \Eg., given two RTEs, we can \emph{intersect} their DFAs to create a
  DFA which will recognize sequences which simultaneously match both
  RTEs.

  From a DFA we can generate an RTE  \code{Extract.dfaToRte(...)}.

  Thus given two RTEs, we can generate an \emph{intersected} RTE with will match
  sequence which simultaneously match both original RTEs.

  In many cases we can determine whether the \emph{language} of an RTE (or DFA)
  is inhabited or empty.

  We can determine whether two RTEs are equivalent by computing the symmetric
  difference of the to corresponding DFAs.  \Ie, are their any sequences in the
  language of one RTE which are not in the language of the other?
\end{frame}
